\documentclass[a4paper,11pt]{article}
\usepackage[top=0in, bottom=1in, left=1in, right=1in]{geometry}

\title{Essayopgave 1\\Linear Regression and Model-building}
\author{}
\date{}

\begin{document}

\maketitle

\noindent {\bfseries Due: 28. February at 12:00 to Thomas via email ({\texttt tleeper@ps.au.dk})}


\vspace{2em}

\noindent {\em Consider the following hypothesis and analysis plan}:

\begin{quote}
Compared to non-democracies, democratic states produce better health outcomes for their citizens. This expectation is tested with a large-n analysis of a cross-section of countries.
\end{quote}

\noindent {\em Your task}:
\begin{enumerate}
\item Write a research question, which could be addressed by the above-stated hypothesis and plan of analysis.
\item Write a short research paper (3--5 pages) that incorporates your research question, the above hypothesis, your operationalizations of ``democracy'' and ``health outcomes,'' the specification of your causal model with a discussion of included variables and how they address concerns about omitted variable bias, and a discussion of your results. You do not need to cite outside literature nor elaborate any theory to justify the provided hypothesis. The paper should include:
\begin{itemize}
\item A brief discussion of the measures
\item Identification of possible confounding variables, which might bias the observed bivariate relationship between X and Y
\item Graphical and/or correlational methods to test whether any of the putative confounding factors should be in the regression model
\item A comparison of your full model to a ``reduced'' model with fewer covariates than the full model (e.g., using {\ttfamily nestreg})
\item A clear interpretation of the size of the relationship (if any) between X and Y
\end{itemize}
\item Include your complete Stata do-file for all included analyses as an appendix to the paper. (Note: Use a fixed-width font (e.g., {\ttfamily Courier}) for this appendix to aid readability.)
\end{enumerate}

\noindent {\em Data}:\\
Visit the website for The Quality of Government Institute at the University of Gothenburg. There you will find the ``Quality of Government Standard Cross-Section Data.''\footnote{A direct link is: {\texttt http://www.qog.pol.gu.se/data/datadownloads/qogstandarddata/}.} You can download the Stata version of the data (.dta file extension) and a codebook describing the data.

\vspace{1em}

\noindent {\em Variables}:\\
The hypothesis includes two constructs: democracy and health outcomes. Neither of these constructs has a single clear operationalization. You will have to choose variables included within the QoG dataset to operationalize each. You may want to consider, for democracy, the Freedom House/Polity index ({\bfseries fh\_polity2}, {\bfseries fh\_ipolity2}) or Economist Intelligence Unit Index of Democracy ({\bfseries eiu\_iod}). For health outcomes, you may want to consider infant mortality (e.g., {\bfseries wdi\_mort}) or life expectancy (e.g., {\bfseries wef\_lifexp}). Justify your choice in any case.


\end{document}