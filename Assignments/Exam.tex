\documentclass[a4paper,12pt]{article}
\usepackage[margin=1in]{geometry}
\usepackage{setspace}
\usepackage{hyperref}

\title{Exam: Quantitative Political Analysis}
\author{}
\date{}

\begin{document}

\maketitle

\vspace{-3em}

\onehalfspacing

\noindent Consider the following research question:

\begin{quote}
What is the effect of economic performance on the quality of democracy?
\end{quote}

\noindent Your assignment should address this question by stating a hypothesized relationship between economic performance and quality of democracy and testing that hypothesis empirically. Your outcome should be measured in two distinct ways: one involving a continuous measure and one involving a dichotomous (binary) measure. You must analyze both operationalizations using appropriate statistical analyses. Your assignment must also include the following elements:

\begin{enumerate}
\item Statement of a hypothesis that is \textit{conditional} in nature (i.e., implying that the effect of economic performance varies across time periods, across countries, or across features of time periods or countries). 
\item Theoretical discussion (one-paragraph) justifying your hypothesis.
\item Description of your independent and dependent variables and the coding thereof.
\item Discussion of your \textit{causal identification strategy} (i.e., what you have done to enable you to interpret the relationship between economic performance and democracy as a causal relationship).
\item Justification of your choice of statistical methods based on theory, data, and relevant statistical tests.
\item Interpretation of the statistical significance and substantive size and meaning of the relationships identified in your analyses.
\end{enumerate}

\noindent For the assignment, you should use the Quality of Government (QoG) Standard Time-Series Data available from \url{http://qog.pol.gu.se/data/datadownloads/qogstandarddata}. A codebook for these data is available on the same web page.


\end{document}
