\documentclass[a4paper,12pt]{article}
\usepackage[top=0in, bottom=1in, left=1in, right=1in]{geometry}

\title{Interpreting Regression Coefficients}
\author{}
\date{}

\begin{document}

\maketitle

For each of the following regression models, interpret the substantive size of the estimated effect. Assume the coefficients are unbiased estimates of relevant population parameters and that all reach conventional levels of statistical significance.

\begin{enumerate}\itemsep0.5em
\item In a regression of countries' military expenditures per capita (measured in U.S. dollars per person) on years since the country's last armed conflict, $\beta_0 = 2000$ and $\beta_1 = -75$.
\item In a regression of individuals' education attainment (measured in years of formal education) on their parents' average education attainment, $\beta_0 = 0$ and $\beta_1 = .8$.

\vspace{1.5em}

\item In a regression of number of years since last civil war on an indicator variable for democracy (where non-democracy = 1 and democracy = 0), $\beta_0 = 110$ and $\beta_1 = -95$.
\item In a regression of national unemployment rates (measured in percent of the working age population) on an indicator variable for the Eurozone (with Euro = 1 and without Euro = 0), $\beta_0 = 12.5$ and $\beta_1 = -12$.

\vspace{1.5em}

\item In a regression of individuals' annual charitable contributions (measured in Euro) on those individuals' self-reported level of concern about poverty (measured on a five point scale from 0 to 4, with 4 meaning ``very concerned''), $\beta_0 = 50$ and $\beta_1 = 4.5$.
\item In a regression of government employees' job satisfaction (measured on a continuous scale from 0 to 100) on a categorical variable recording the ministry each works for, $\beta_0 = 68$ and $\beta_1 = -5$.

\vspace{1.5em}

\item In a regression of lifetime earnings (i.e., lifetime cumulative income, measured in millions of kr) on indicators for having a degree in political science ($x_1$) and a degree in economics ($x_2$), $\beta_0 = 8.5$, $\beta_1 = 4.2$, and $\beta_2 = 4.7$.
\item In a regression of voter turnout rates (i.e., percentage voter turnout) across municipalities on population size ($x_1$) and median household income ($x_2$, measured in thousands of kr), $\beta_0 = 35$, $\beta_1 = -.001$, and $\beta_2 = .001$.

\end{enumerate}

\end{document}