\documentclass[a4paper,12pt]{article}
\usepackage{hyperref}
\usepackage[top=1in, bottom=1in, left=1in, right=1in]{geometry}

\begin{document}

\begin{center}
\textbf{Instrumental Variables Activity}
\end{center}

\noindent Your task is to read each example below and then address the following:
\begin{itemize}
\item Is the instrument credibly exogenous? Why or why not?
\item Is the instrument likely to be strong (has a large causal effect on the endogenous variable) or weak (has a small causal effect on the endogenous variable)?
\end{itemize}

\noindent Valid instruments are those that are (1) relevant and (2) credibly exogenous. A {\em relevant} instrument is causally related to an endogenous explanatory variable. An {\em credible} or {\em exogenous} instrument is unrelated to any other factors that cause the endogenous variable or any other factors that cause the outcome. We can empirically test whether an instrument is relevant --- strong instruments have large causal effects and weak instruments have small causal effects on the endogenous variable --- but we cannot empirically test whether an instrument is credible.


\begin{enumerate}\itemsep0.5em

\item What is the effect of past military service on labor-market earnings? Military service is confounded by many things. A possible instrument is a wartime draft (conscription) lottery, where some individuals are randomly assigned to be drafted. But, some people serve in the military without being drafted and some people who are randomly called to be drafted end up not serving. % Angrist (1990)

\item What is the effect of smoking during pregnancy on infant birth weight? Smoking and birth weight share common causes (e.g., maternal income). A possible instrument is regional variation in cigarette taxation rates. % Evans and Ringel (1999)

\item What is the effect of economic growth on civil wars in Africa? Economic growth is caused by many things, including factors that also explain the outbreak of civil war. Year-to-year variations in rainfall is a possible instrument for economic growth. % Miguel, Satyanath, and Sergenti (2004)

\item What is the effect of viewing a political party's televised campaign presentation program on an individual's support for that party? Post-exposure party support and willingness to view the program are possibly confounded by many factors. A possible instrument for viewing the program is being randomly encouraged to view the program by researchers conducting a pre-post survey. % Albertson and Lawrence (2009)

\item What effect do institutions have on economic growth? The form and quality of government institutions is confounded by a number of factors that also explain growth. A possible instrument in countries that were formally colonies is the mortality rate of early colonial settlers. % Acemoglu, Johnson, and Robinson (2001)

\item What effect do dams have on poverty levels? The construction of dams is confounded by other factors like access to electricity and drinking water explain both dam construction and poverty levels. A possible instrument is the gradient (or slope) of rivers, which determine suitability of a particular section of a river for the construction of a dam. % Duflo and Pande (2007)

\item What effect did West German television have on the political support of East German citizens for the East German regime? A possible instrument is distance individuals lived from Dresden (in far eastern Germany), which increased individuals' access to West German television. % Kern and Hainmueller (2009)

\item What effect does political candidate spending have on election outcomes? Spending in the current campaign is confounded by factors that both predict spending and outcomes, like candidate quality, political positions, and economy performance. A possible instrument is spending in the previous campaign (or any previous campaign). %Gerber (1998)

\item What effect does unemployment have on crime rates? Unemployment and crime are both caused by many common factors. A possible instrument is the closure of military bases, which might affect a local area's unemployment rate. % Raphael and Winter-Ebmer (2000)

\item What effect does the size of police forces have on crime rates? The number of police in an area and the crime rate in the same area are confounded by many third variables. A possible instrument (in an over-time analysis) is the timing of elections because the sizes of police forces might fluctuate with electoral cycles. % Levitt (1997)

\item What is the effect of university education on labor market earnings? A possible instrument is the distance an individual lived during high school from the nearest university. % Card (1995)

\item What is the effect of primary school class sizes on student learning? Class sizes and educational performance are possibly commonly caused by many things. A possible instrument is a government policy restricting maximum class sizes, such that when the number of students in a class exceeds a threshold the class is split in two. % Angrist and Lavy (1999)



\end{enumerate}



\end{document}