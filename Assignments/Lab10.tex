\documentclass[a4paper,12pt]{article}
\usepackage[top=1in, bottom=1in, left=1in, right=1in]{geometry}

\begin{document}

\begin{center}
\textbf{GLM Interpretation Lab}
\end{center}

\noindent Please briefly read the following to get a sense of the article's main empirical analysis:

\begin{quote}
Houston, David J. 2005. ```Walking the Walk' of Public Service Motivation: Public Employees and Charitable Gifts of Time, Blood, and Money.'' {\em Journal of Public Administration Research and Theory} 16: 67--86. doi:10.1093/jopart/mui028
\end{quote}

\noindent The data for this article come from the General Social Survey, a nationally representative survey of the United States population. These data, in Stata (.dta) format, are available to you on Blackboard in the \texttt{Content > GSS} folder, along with the output of Stata's \texttt{codebook} command in a separate .txt file.

\begin{enumerate}\itemsep1em
\item Load the data file into Stata.
\item Start a new do-file to contain the following analyses.
\item Create dummy variables for each of the three outcomes:
	\begin{itemize}
	\item Recode \texttt{volchrty} as a binary variable called \texttt{volbin} 
	\item Recode \texttt{givblood} as a binary variable called \texttt{bloodbin}
	\item Recode \texttt{givchrty} as a binary variable called \texttt{charbin}
	\end{itemize}
\item Generate a new variable that recodes the key causal variable \texttt{wrkgovt} to be 1 for public sector employees and 0 for private sector employees. Call this new variable \texttt{pubemp}.
\item Recode or generate covariates as necessary based on th discussion on p.74 of Houston (2005). You will need versions of the following variables:
	\begin{itemize}
	\item Female: \texttt{sex}
	\item White: \texttt{race}
	\item Education: \texttt{educ}
	\item Income: \texttt{rincom98}
	\item Occupational prestige: \texttt{prestg80}
	\item Married: \texttt{marital}
	\item Age: \texttt{age}
	\item Children in household: a sum of \texttt{babies}, \texttt{preteen}, and \texttt{teens}
	\item Community size (logged): \texttt{size}
	\item Church attendance: \texttt{attend}
	\end{itemize}
\item Replicate the analyses in Tables 3, 4, and 5 of Houston (2005). Focus on Model \# 1 in each table, using the \texttt{logit} command to perform logistic regression. It is used in the same way as \texttt{reg}.
\item Re-estimate the same models using \texttt{probit} instead of \texttt{logit} to retrieve probit regression results.
\item Use \texttt{eststo} and \texttt{esttab} to display the results of both models side-by-side.
%\item Install the \texttt{prvalue} program by typing \texttt{findit spostado} into Stata and following the on-screen instructions to install \texttt{install spost9\_ado}.
% Goodness of fit statistics \texttt{fitstat}
\item Use \texttt{tab pubemp volbin, row} to see the proportions of public and private sector employees who volunteered time in the sample. Repeat this for the other outcomes \texttt{bloodbin} and \texttt{charbin}.
\item Estimate the predicted probability of each outcome for public and private sector employees using \texttt{margins, at(pubemp=(0 1))}. {\em Remember to call \texttt{margins} immediately after \texttt{logit}.}
\item You can simplify the \texttt{margins} call by treating \texttt{pubemp} as a factor when you call \texttt{logit}. You can then just type \texttt{margins pubemp}. Try it. Manually calculate the average marginal effect by calculating the difference between the two predicted probabilities.
\item Estimate the predicted probability of the outcome for public and private sector employees across levels of education using: \texttt{quietly margins pubemp, at(educ=(0 (1) 20))}.
%Compare the empirical probabilities (i.e., the sample proportions) reported in Table 1 to predicted probabilities from the logit model. What is the predicted probability of government and non-governmental employees engaging in each of the three types of voluntary behavior?
\item Plot the previous result using \texttt{marginsplot}. Because predicted probabilities range from 0 to 1, it is best to specify the scale of the y-axis. Try it again using \texttt{marginsplot, ylabel(0 (0.1) 1)}.
\item Estimate the average marginal effect of being a public employee using \texttt{margins, dydx(pubemp)}. How does this compare to the manually calculated change in predicted probability you calculated earlier?
\item Compare the average marginal effect to the marginal effect at the means of the other covariates using: \texttt{margins, dydx(pubemp) atmeans}. How do they differ? Why?
\item Estimate the predicted probabilities of the outcome across different age levels and plot the results. You can use \texttt{quietly} to suppress the printing of the \texttt{margins} command before you plot. Try the following code:\\
\texttt{margins pubemp, at(age=(20 (5) 80))\\
marginsplot}\\
How large does the marginal effect appear to be at each age level?
\item Now let Stata figure out the size of the marginal effects for you and plot the result. Try the following code:\\
\texttt{margins, dydx(pubemp) at(age=(20 (5) 80))\\
marginsplot}\\
How do the last two graphs compare?
\end{enumerate}

\end{document}