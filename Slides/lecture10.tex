\documentclass[14pt]{beamer} %Makes presentation
%\documentclass[handout]{beamer} %Makes Handouts
\usetheme{Singapore} %Gray with fade at top
\useoutertheme[subsection=false]{miniframes} %Supppress subsection in header
\useinnertheme{rectangles} %Itemize/Enumerate boxes
\usecolortheme{seagull} %Color theme
\usecolortheme{rose} %Inner color theme

\definecolor{light-gray}{gray}{0.75}
\definecolor{dark-gray}{gray}{0.55}
\setbeamercolor{item}{fg=light-gray}
\setbeamercolor{enumerate item}{fg=dark-gray}

\setbeamertemplate{navigation symbols}{}
%\setbeamertemplate{mini frames}[default]
\setbeamercovered{dynamics}
\setbeamerfont*{title}{size=\Large,series=\bfseries}

%\setbeameroption{notes on second screen} %Dual-Screen Notes
%\setbeameroption{show only notes} %Notes Output

\setbeamertemplate{frametitle}{\vspace{.5em}\bfseries\insertframetitle}
\newcommand{\heading}[1]{\noindent \textbf{#1}\\ \vspace{1em}}

\usepackage{bbding,color,multirow,times,ccaption,tabularx,graphicx,verbatim,booktabs,fixltx2e}
\usepackage{colortbl} %Table overlays
\usepackage[english]{babel}
\usepackage[latin1]{inputenc}
\usepackage[T1]{fontenc}
\usepackage{lmodern}

%\author[]{Thomas J. Leeper}
\institute[]{
  \inst{}%
  Department of Political Science and Government\\Aarhus University
}

\usepackage{tikz}
\usetikzlibrary{shapes,arrows}
\usetikzlibrary{decorations.pathreplacing}

\title{Interpretation of GLMs}

\date[]{April 21, 2015}

\begin{document}

\frame{\titlepage}

\frame{\tableofcontents}

\section{Essay 2}


% causal graphs vs. linear path models vs. ``heuristic'' theoretical drawings




% graphing and interpretation activity

% we are estimating coefficients for an equation

% using estimates, we can create predicted or fitted values based on input values

% graph these predicted values





\section{Review GLMs and MLE}


% ways of thinking about this

% transforming outcome to a continuous scale so that we can estimate effects

% latent utility

% latent variable



\frame{

\frametitle{The Likelihood Function}
 
 - Takes possible population parameters as inputs
   - i.e., guess at `\(\boldsymbol{\beta}\)`
   
 - Give a *likelihood* of seeing our sample data given that input
   - i.e., the probability of observing the data given the parameter guesses
 
 - We pick best guess
   - i.e., the guesses at `\(\boldsymbol{\beta}\)` that yield the *maximum* likelihood

}


\frame{

\frametitle{Maximizing the likelihood}

How do we find the maximum likelihood (and thus our estimated `\(\beta\)`'s)?

We guess! Repeatedly!
  
  - Start with some kind of guess and calculate the likelihood
  
  - Make better and better guesses, until the likelihood doesn't get marginally higher

}




% assumptions




\section{Interpreting GLMs}

% activity discussing one of the assigned articles (Hobolt maybe?)




% coefficients in logit and probit

% average marginal effects
% marginal effects at the mean
% marginal effects at representative values

% latent scale versus probability scale

\frame{

What is the probability that $y_i=1|\boldsymbol{x}_i$ ?

}


% marginal effects (slope of the response curve)
% discrete changes
%% there is no difference in linear additive (i.e., many OLS) models because the effect of X is constant across levels of X



% graphing predicted probabilities
% graphing marginal effects



% interaction terms


\section{Model Specification}

% logit versus probit

% binary, categorical, ordered outcomes

% robust SEs - don't necessarily make sense in a GLM context


% elaborating beyond binary outcomes (ordered, categorical, count)



\appendix
\frame{}

\end{document}
