\documentclass[14pt]{beamer} %Makes presentation
%\documentclass[handout]{beamer} %Makes Handouts
\usetheme{Singapore} %Gray with fade at top
\useoutertheme[subsection=false]{miniframes} %Supppress subsection in header
\useinnertheme{rectangles} %Itemize/Enumerate boxes
\usecolortheme{seagull} %Color theme
\usecolortheme{rose} %Inner color theme

\definecolor{light-gray}{gray}{0.75}
\definecolor{dark-gray}{gray}{0.55}
\setbeamercolor{item}{fg=light-gray}
\setbeamercolor{enumerate item}{fg=dark-gray}

\setbeamertemplate{navigation symbols}{}
%\setbeamertemplate{mini frames}[default]
\setbeamercovered{dynamics}
\setbeamerfont*{title}{size=\Large,series=\bfseries}

%\setbeameroption{notes on second screen} %Dual-Screen Notes
%\setbeameroption{show only notes} %Notes Output

\setbeamertemplate{frametitle}{\vspace{.5em}\bfseries\insertframetitle}
\newcommand{\heading}[1]{\noindent \textbf{#1}\\ \vspace{1em}}

\usepackage{bbding,color,multirow,times,ccaption,tabularx,graphicx,verbatim,booktabs,fixltx2e}
\usepackage{colortbl} %Table overlays
\usepackage[english]{babel}
\usepackage[latin1]{inputenc}
\usepackage[T1]{fontenc}
\usepackage{lmodern}

%\author[]{Thomas J. Leeper}
\institute[]{
  \inst{}%
  Department of Political Science and Government\\Aarhus University
}

\usepackage{tikz}
\usetikzlibrary{shapes,arrows}
%\usetikzlibrary{decorations.pathreplacing}

\title{Interpretation of GLMs}

\date[]{April 21, 2015}

\begin{document}

\frame{\titlepage}

\frame{\tableofcontents}

\section{Essay 2}


\frame{
    \frametitle{Feedback on Essay 2}
    \begin{itemize}\itemsep1em
    \item Overall, these were ``OK'' but not great
    \item Issues mainly related to interpretation
    \item Other issues outlined in the email I sent
    \item Review a few of these issues quickly
    \end{itemize}
}

\begin{frame}[fragile]
    \frametitle{A ``Heuristic'' Graph}
	\begin{center}
	\begin{tikzpicture}[>=latex',circ/.style={draw, shape=circle, node distance=5cm, line width=1.5pt}]
        \draw[->] (0,0) node[left] (X) {X} -- (5,0) node[right] (Y) {Y};
        \draw[-] (X) -- (2.5,0) node[right] (D) {};
        \draw[->] (3.1,0) -- (5,0) (Y);
        \draw[->] (-2,2) node[left] (Z) {Z} -- (Y);
        \draw[->] (Z) -- (X);
        \draw[->] (2.5,-1) node[below] (M) {M} -- (D);
    \end{tikzpicture}
    \end{center}
\end{frame}
% causal graphs vs. linear path models vs. ``heuristic'' theoretical drawings

\begin{frame}<0>[fragile]
    \frametitle{Better: Two Causal Graphs}
	\begin{center}
	\begin{tikzpicture}[>=latex',circ/.style={draw, shape=circle, node distance=5cm, line width=1.5pt}]
        \draw[->] (0,0) node[left] (X1) {X} -- (2,0) node[right] (Y1) {Y};
        \draw[->] (-1,2) node[left] (Z1) {Z} -- (Y1);
        \draw[->] (Z1) -- (X1);
        \draw[->] (0,-2) node[left] (M1) {M} -- (Y1);

        \draw[->] (5,0) node[left] (X2) {X} -- (7,0) node[right] (Y2) {Y};
        \draw[->] (4,2) node[left] (Z2) {Z} -- (Y2);
        \draw[->] (Z2) -- (X2);
        \draw<1>[->] (5,-2) node[left] (M2) {M} -- (Y2);
    \end{tikzpicture}
    \end{center}
\end{frame}



\begin{frame}[fragile]
    \frametitle{Representing Theory as Graph}
	\begin{center}
	\begin{tikzpicture}[>=latex',circ/.style={draw, shape=circle, node distance=5cm, line width=1.5pt}]
        \draw[->] (0,0) node[left] (X) {X} -- (4,0) node[right] (Y) {Y};
        \draw[-] (X) -- (2.5,0) node[right] (D) {};
        \draw[->] (3.1,0) -- (Y);
        \draw[->] (-1,2) node[left] (Z) {Z} -- (Y);
        \draw[->] (Z) -- (X);
        \draw<2->[->] (0,-2) node[left] (M) {M} -- (Y);
        \draw<3->[->] (-3,-1) node[left] (W) {W} -- (M);
        \draw<3->[->] (W) -- (X);
    \end{tikzpicture}
    \end{center}
\end{frame}


\frame{
    \frametitle{Interpretation}
    \begin{itemize}\itemsep1em
    \item<1-> How do we interpret a coefficient estimate in a regression model with no interaction terms?
    \item<2-> How do we interpret a coefficient estimate in a regression model with one or more interaction terms?
    \item<3-> How do we calculate predicted or fitted values from our estimates?
    \item<4-> How can we visually display the results of a regression estimation?
    \end{itemize}
}

% we are estimating coefficients for an equation

% using estimates, we can create predicted or fitted values based on input values

% graph these predicted values


\frame{
    \frametitle{Graphing Activity}
    \begin{itemize}\itemsep1em
    \item It can be difficult to see regression without results without graphing them
    \item But, using Stata it can be unclear how the estimated regression equation conforms to particular graphical output
    \item So, you will practice graphing \textit{by hand}
    \item Do activities \# 1--3
    \end{itemize}
}









\section{Review GLMs and MLE}


\frame{
    \frametitle{Non-continuous Outcomes}
    \begin{enumerate}\itemsep1em
    \item Why shouldn't we use OLS for a non-continuous outcome variable?
    \item<2-> What do we do instead? 
    \end{enumerate}
}


\frame{
    \frametitle{Regression on a Latent Variable}
    \begin{itemize}\itemsep1em
    \item Consider a binary outcome $y$ (e.g., voting)
    \item OLS provides a nonsensical fit to the outcome
    \item Think about the problem as a ``latent'' outcome ($y\ast$) that manifests in two observed categories
        \begin{itemize}
        \item As $y\ast$ increases, $Pr(Y=1) \rightarrow 1$
        \item As $y\ast$ decreases, $Pr(Y=1) \rightarrow 0$
        \end{itemize}
    \item We do not observe $y\ast$, only $y$
    \end{itemize}
}

\frame{
    \frametitle{Estimation in GLM}
    \begin{itemize}\itemsep1em
    \item In OLS, we estimate: $\hat{y} = \beta_0 + \beta_1 x + e$
    \item This represents the conditional mean of $y$
    \item In a GLM, we estimate: $\hat{y}\ast = \beta_0 + \beta_1 x + e$\\
    where $y\ast$ is a transformation of $y$
    \item This is also a prediction of the conditional mean of $y$
    \item<2-> How do we transform $y$ to $y\ast$?
    \end{itemize}
}

\frame{
    \frametitle{Model Specification}
    \begin{enumerate}\itemsep1em
    \item<1-> Complete set of conditioning variables
    \item<2-> Correctly specified model
    \item<3-> Choice of error distribution
    \item<4-> Link function
    \end{enumerate}
}

\frame{
	\frametitle{Error Distribution}
	\begin{itemize}\itemsep1em
	\item 
	\end{itemize}
}


\frame{
    \frametitle{Link Function}
    \begin{itemize}\itemsep2em
    \item $y\ast = X\beta = \beta_0 + \beta_1 x_1 + \beta_2 x_2 + \dots$
    \item \textbf{Link function}: $g(\mu) = X\beta$
        \begin{itemize}
        \item Transforms $y$ to $y\ast$
        \end{itemize}
    \item \textbf{Inverse link function}: $\mu = g^{-1}(X\beta)$
        \begin{itemize}
        \item Transforms $y\ast$ back to $y$
        \end{itemize}
    \end{itemize}
}

\frame{
	\frametitle{Choosing a Link Function}
	\begin{itemize}\itemsep1em
	\item Based on expected distribution of the error term of $y\ast$
	\item Choice heavily influenced by convention rather than empirics
	\item Choice of link adds \textit{model dependence}!
		\begin{itemize}
		\item Expected influence of $x$ on $y$ now depends on choice of link
		\item Different link function can yield different substantive and statistical results
		\end{itemize}
	\end{itemize}
}

\frame{
    \frametitle{Common Link Functions}
    \begin{itemize}\itemsep1em
    \item Identity
    \item Logit
    \item Probit
    \item Poisson
    \end{itemize}
}

% plot of Normal and logit distribution


\frame{
	\frametitle{Logit vs. Probit}
	\begin{itemize}\itemsep1em
	\item<1-> Both constrain a continuous $y\ast$ to (0,1)
	\item<2-> Probabilities are symmetric
	\item<3-> Logit allows us to estimate odds-ratios
	\item<4-> Logit is maybe slightly more common in political science for what are probably just historical reasons
	\end{itemize}
}

% logit and probit both introduce a symmetrical relationship between covariates and the probability that y = 1
% other models do not assume that symmetry (log-log; complementary log-log; etc.)





% elaborating beyond binary outcomes (ordered, categorical, count)






\frame{
    \frametitle{The Likelihood Function}
    \begin{itemize}\itemsep1em
    \item Takes possible population parameters as inputs
    \item Give a \textit{probability} of seeing each observation in our sample data given that input parameters
    \item Combine those probabilities (i.e., likelihoods)
        \begin{itemize}
        \item Multiply the likelihoods
        \item Add the log-likelihoods
        \end{itemize}
    \item Repeat process for various parameter values
    \item We pick the best guess from all of those that we test
    \end{itemize}
}



% assumptions




\section{Interpreting GLMs}

\frame{
    \frametitle{Interpretation}
    \begin{itemize}\itemsep1em
    \item Recall the Hobolt article from last week
    \item What is her research question? What is her theory?
    \item What is the test of that theory?
    \item How does she interpret the results?
    \end{itemize}

}

% coefficients in logit and probit

\frame{
    \frametitle{Coefficients}
    \begin{itemize}\itemsep1em
    \item Coefficients express effect of $x$ on $y\ast$
    \item In logistic regression, this is a statement about the odds-ratio:
        $\hat{\beta} = \dfrac{\frac{p_1}{1-p_1}}{\frac{p_0}{1-p_0}}$
    \item Coefficients are hard to interpret \textit{substantively}
    \item Statistical significance is similar to OLS
    \end{itemize}
}

\frame{
    \frametitle{Predicted Outcomes}
	\begin{itemize}\itemsep1em
	\item In OLS, fitted values from the estimated regression equation are values of $y$
	\item In most GLMs, fitted values are expressed for $y\ast$
	\item To interpret logit or probit, we transform to predicted probabilities
	\item \textbf{Predicted probability}: What is the probability that $y_i=1|\boldsymbol{x}_i$?
	\end{itemize}
}

\frame{
    \frametitle{Predicted Probabilities}
    \begin{itemize}
    \item 
    \end{itemize}
}



\frame{
    \frametitle{Marginal Effects}

}


\frame{
    \frametitle{Graphing Activity (cont.)}
    \begin{itemize}\itemsep1em
    \item Do graphing activity \# 4
    \end{itemize}
}

\frame{
    \frametitle{Three Types of Marginal Effects}
    \begin{enumerate}\itemsep1em
    \item Marginal Effects at the Mean (MEMs)
    \item Marginal Effects at Representative Values (MERs)
    \item Average Marginal Effects (AMEs)
    \end{enumerate}
}


\frame{
    \frametitle{MEM}
    \begin{itemize}\itemsep1em
    \item 
    \end{itemize}
}

\frame{
    \frametitle{MERs}
    \begin{itemize}\itemsep1em
    \item
    \end{itemize}
}

\frame{
    \frametitle{AMEs}
    \begin{itemize}\itemsep1em
    \item The MEM/AME distinction
    \end{itemize}
}


\frame{
    \frametitle{Discrete Changes}
    \begin{itemize}\itemsep1em
    \item Marginal effects are \textit{instantaneous changes}
    \item This makes sense for continuous independent variables
    \item For categorical (factor) variables, we often instead calculate a discrete change
        \begin{itemize}
        \item $Pr(y=1|x=1) - Pr(Y=1|x=0)$
        \item Marginal effect and discrete change are the same in OLS
        \end{itemize}
    \end{itemize}
}

\frame{
    \frametitle{Graphing Activity (cont.)}
    \begin{itemize}\itemsep1em
    \item Do graphing activity \# 5
    \end{itemize}
}


\frame{
    \frametitle{Interaction Terms}
    \begin{itemize}\itemsep1em
    \item Due to the link function transformation, the marginal effect of $x$ depends on the value of $x$ and all other covariates
    \item This creates \textit{implicit} interactions
    \item We still have to include \textit{explicit} interaction terms to estimate heterogeneous effects (i.e., effect moderation)
    \end{itemize}
}


\frame{
    \frametitle{Graphing Activity (cont.)}
    \begin{itemize}\itemsep1em
    \item Do graphing activity \# 6
    \end{itemize}
}





\appendix
\frame{}

\end{document}
