\documentclass[14pt]{beamer} %Makes presentation
%\documentclass[handout]{beamer} %Makes Handouts
\usetheme{Singapore} %Gray with fade at top
\useoutertheme[subsection=false]{miniframes} %Supppress subsection in header
\useinnertheme{rectangles} %Itemize/Enumerate boxes
\usecolortheme{seagull} %Color theme
\usecolortheme{rose} %Inner color theme

\definecolor{light-gray}{gray}{0.75}
\definecolor{dark-gray}{gray}{0.55}
\setbeamercolor{item}{fg=light-gray}
\setbeamercolor{enumerate item}{fg=dark-gray}

\setbeamertemplate{navigation symbols}{}
%\setbeamertemplate{mini frames}[default]
\setbeamercovered{dynamics}
\setbeamerfont*{title}{size=\Large,series=\bfseries}

%\setbeameroption{notes on second screen} %Dual-Screen Notes
%\setbeameroption{show only notes} %Notes Output

\setbeamertemplate{frametitle}{\vspace{.5em}\bfseries\insertframetitle}
\newcommand{\heading}[1]{\noindent \textbf{#1}\\ \vspace{1em}}

\usepackage{bbding,color,multirow,times,ccaption,tabularx,graphicx,verbatim,booktabs,fixltx2e}
\usepackage{colortbl} %Table overlays
\usepackage[english]{babel}
\usepackage[latin1]{inputenc}
\usepackage[T1]{fontenc}
\usepackage{lmodern}

%\author[]{Thomas J. Leeper}
\institute[]{
  \inst{}%
  Department of Political Science and Government\\Aarhus University
}

\usepackage{tikz}
\usetikzlibrary{shapes,arrows}
%\usetikzlibrary{decorations.pathreplacing}

\title{Panel GLMs}

\date[]{May 12, 2015}

\begin{document}

\frame{\titlepage}

\frame{\tableofcontents}


\section{Review of Panel Data}
\frame{\tableofcontents[currentsection]}

\frame{
    \frametitle{Segue From Event-History Analysis}
    \begin{itemize}\itemsep1em
    \item Event history analysis involves the analysis of durations and probabilities of state changes over time across many units
    \item Each unit's trajectory or history can begin at an arbitrary point in time
        \begin{itemize}
        \item Ex. 1: Colony's time to independence after 1900
        \item Ex. 2: Durability of democratic government after independence
        \end{itemize}
    \item In problems (like Ex. 1), we are interested in studying units over the \textit{same period of time}
    \end{itemize}
}

\frame{
    \frametitle{Panel Analysis}
    \begin{itemize}\itemsep1em
    \item In event history analysis, time is our key variable
    \item In panel analysis:
        \begin{itemize}
        \item unit characteristics are our key variables
        \item observations exist simultaneously
        \end{itemize}
    \item We are interested in effects of changes in $X$ on $Y$
    \end{itemize}
}


% terminology
\frame{
    \frametitle{Terminology}
    \begin{itemize}\itemsep1em
    \item<2-> \textit{Panel}
    \item<3-> \textit{Wide} versus \textit{Long} data
    \item<4-> \textit{Time-varying} versus \textit{time-invariant}
    \item<5-> \textit{Balanced} versus \textit{Unbalanced} panel
    \item<6-> \textit{Fixed effects}
    \item<7-> \textit{Random effects}
    \end{itemize}
}


\frame{
    \frametitle{Panel versus Time-Series}
    \begin{itemize}\itemsep1em
    \item Cross-sectional data involve many units observed at one time
    \item Panel data involve many units over at multiple points in time
    \item Time-series data involve one (or more) units observed at multiple points time
    \item Time-Series, Cross-Sectional (TSCS) data are panel data
        \begin{itemize}
        \item Sometimes the units are aggregations
        \end{itemize}
    \item \textit{Within-subjects} is also a synonym for panel analysis
    \end{itemize}
}


\frame{
    \frametitle{Causal Inference}
    \begin{itemize}\itemsep1em
    \item What is the goal of causal inference?
    \item How do we define a causal effect (in terms of counterfactuals)?
    \vspace{1em}
    \item<2-> If $X_i$ is time-varying, we observe $Y_i$ for the same unit $i$ when $X_i$ takes on different values
    \item<3-> Is this the same as observing both $Y_0it$ and $Y_1it$?
    \item<4-> Then why are panel data useful?
    \end{itemize}
}



\section{Model Types}
\frame{\tableofcontents[currentsection]}

\frame{
    \frametitle{Nonlinear Panel Models Examples}
    \begin{itemize}
    \item<2-> Binary outcome
    \item<3-> Ordered outcome
    \item<4-> Count outcome
    \item<5-> Multinomial outcome
    \item<6-> Censored
    \end{itemize}
}


\frame{
    \frametitle{Research Questions}
    \begin{itemize}\itemsep1em
    \item Form groups of 4
    \item Generate a research question involving:
        \begin{itemize}
        \item Binary outcome
        \item Ordered outcome
        \item Count outcome
        \end{itemize}
    \item For each type, generate an institutional- and an individual-level question
    \item So 6 research questions total
    \end{itemize}
}




\frame{
    \frametitle{Review: Basic Panel Approaches}
    \begin{itemize}\itemsep2em
    \item Pooled estimator
    \item Fixed effects estimator
    \item Random effects estimator
    \end{itemize}
}

\frame{
    \frametitle{Estimation Issues}
    \begin{itemize}
    \item Cross-sectional OLS models are easy to estimate
    \item Linear panel models are fairly easy to estimate
    \item<2-> Cross-sectional GLMs are hard to estimate
        \begin{itemize}
        \item No closed-form solution
        \item Often rely on maximization algorithms
        \end{itemize}
    \item<3-> Nonlinear panel models are also hard to estimate
    \end{itemize}
}

\frame{
    \frametitle{Who cares?}
    \begin{itemize}\itemsep2em
    \item If Stata can give us numbers, who cares what's happening?
    \item More difficult problem means greater diversity of solutions
        \begin{itemize}
        \item No obvious best solution
        \item Terminology overload
        \item Assumptions!
        \end{itemize}
    \item<2-> Be cautious when treading into unfamiliar waters
    \end{itemize}
}

\frame{
    \frametitle{Terms You Will See}
    \begin{itemize}
    \item Quadrature
    \item Conditional Likelihood
    \item Simulated Likelihood
    \item Generalized Estimating Equation (GEE)
    \item Generalized Method of Moments (GMM)
    \end{itemize}
}

\frame{\frametitle{Questions?}}




\frame{
    \frametitle{Pooled Estimator}
    \begin{itemize}\itemsep1em
    \item 
    \end{itemize}
}

\frame{
    \frametitle{Fixed Effects Estimator}
    \begin{itemize}\itemsep1em
    \item 
    \end{itemize}
}

\frame{
    \frametitle{Random Effects Estimator}
    \begin{itemize}\itemsep1em
    \item 
    \end{itemize}
}






% binary (logit/probit)
%% Wawro article


% ordered (logit/prboit)


% count models
%% Seeberg article






\frame{\frametitle{Questions?}}



\section[Next Steps]{Review and Looking Forward}
\frame{\tableofcontents[currentsection]}

\frame{
    \frametitle{Where have we been?}
    \begin{itemize}\itemsep2em
    \item What have we learned in this course?
    \item What haven't we learned in this course?
    \end{itemize}
}

\frame{
    \frametitle{What have we learned?}
    \begin{itemize}\itemsep1em
    \item<2-> Thinking about causality as counterfactuals
    \item<2-> How to obtain causal inference from observational data
    \item<2-> Analyzing continuous outcome data
    \item<2-> Analyzing binary, ordered, and count outcome data
    \item<2-> Analyzing event histories
    \item<2-> Analyzing data over time
    \item<2-> Managing complex data structures
    \item<2-> Data interpretation!
    \end{itemize}
}

\frame{
    \frametitle{What should I learn next?}
    \begin{itemize}
    \item<2-> Measurement: factor analysis, principal components, IRT
    \item<3-> Design: surveys, experiments, data gathering
    \item<4-> Classification: regression trees, classifiers, SVM
    \item<5-> Clustering: K-means, hierarchical clustering
    \item<6-> Nonparametric statistics
    \item<7-> Bayesian statistics
    \item<8-> Time series analysis
    \item<9-> Data visualization
    \item<10-> ``Big data''
    \end{itemize}
}


\frame{
    \frametitle{Goals for this course}
    \begin{itemize}
    \item Describe politically relevant research questions and hypotheses
    \item Evaluate and deduce observable implications from political science theories 
    \item Explain statistical procedures and their appropriate usages
    \item Apply statistical procedures to relevant research problems
    \item Synthesize results from statistical analyses into well-written and well-structured essays
    \item Demonstrate how to use Stata for statistical analysis
    \end{itemize}
}

\frame{
    \frametitle{Exam}
    \begin{itemize}\itemsep1em
    \item Standard 7-day home assignment
    \item We will give you a question and data
    \item You write an essay that answers that question
    \item To do well:
        \begin{itemize}
        \item Understand your analysis
        \item Justify your analysis
        \item Interpret your analysis
        \end{itemize}
    \item Exam allows for considerable flexibility
    \end{itemize}
}

\frame{\frametitle{Questions?}}


\section{}

\frame{
    \frametitle{Course Evaluations}
    \begin{itemize}\itemsep2em
    \item What went well in this course?
    \item What would you like to have gone differently?
    \vspace{1em}
    \item<2-> \url{http://www.survey-xact.dk/LinkCollector?key=YAV25A9Q359N}
    \end{itemize}
}

\frame{
    \frametitle{Preview}
    \begin{itemize}\itemsep2em
    \item Tomorrow: More panel GLMs in Stata
        \begin{itemize}\itemsep1em
        \item Predicted probabilities
        \item Marginal effects
        \item Graphing
        \end{itemize}
    \item Next week: Optional Q/A Session
    \item Readings: Test your knowledge on complex articles
    \end{itemize}
}

\appendix
\frame{}

\end{document}
