\documentclass[11pt,a4paper]{article}
\usepackage[top=1in, bottom=1in, left=1in, right=1in]{geometry}
\usepackage{natbib}

\usepackage{setspace,mdwlist,comment}
\setlength{\marginparwidth}{.5in}
\setlength{\parindent}{0in}

% mini table of contents
\usepackage{minitoc}
\dosecttoc % make section toc
\setcounter{secttocdepth}{2} % subsection depth
\renewcommand{\stctitle}{} % no title
\nostcpagenumbers

\usepackage{natbib}

% Two lines to create in-text full citations for a syllabus
% And comment out my other standard bibtex commands
\usepackage{bibentry}
\newcommand{\reading}[2][]{\noindent -- {#1}\bibentry{#2}.\vspace{.25em}\\}
\newcommand{\textbook}[2][]{\noindent -- {#1}#2.\vspace{.25em}\\}
\newcommand{\thomas}{\vspace{1em}\noindent Instructor: Thomas\\}
\newcommand{\david}{\vspace{1em}\noindent Instructor: David\\}
\newcommand{\seealso}{\noindent \emph{See Also:}}
\newcommand{\topic}[1]{\noindent \textbf{#1}\\}

\usepackage{hyperref}
\hypersetup{
    bookmarks=true,         % show bookmarks bar?
    unicode=false,          % non-Latin characters in Acrobat’s bookmarks
    pdftoolbar=true,        % show Acrobat’s toolbar?
    pdfmenubar=true,        % show Acrobat’s menu?
    pdffitwindow=false,     % window fit to page when opened
    pdfstartview={FitH},    % fits the width of the page to the window
    pdftitle={Syllabus: Quantitative Political Analysis},    % title
    pdfauthor={David Hendry and Thomas J. Leeper},     % author
    pdfsubject={Political Science},   % subject of the document
    pdfnewwindow=true,      % links in new window
    pdfborder={0 0 0}
}

\title{Syllabus for \textit{Quantitative Political Analysis}\\Department of Political Science and Government\\Aarhus University\\Spring Term 2015}
\author{David Hendry\thanks{Office 1341-124, \href{mailto:david.hendry@ps.au.dk}{david.hendry@ps.au.dk}} \and
Thomas J. Leeper\thanks{Office 1340-232, \href{mailto:tleeper@ps.au.dk}{tleeper@ps.au.dk}}}

\begin{document}
\nobibliography*

\maketitle

\faketableofcontents

%\section{Introduction}

Individuals working in the public and private sectors are often tasked with analyzing quantitative data or making use of analyses performed by others. The purpose of this course is to expand and improve significantly participants' ability to perform quantitative analyses of political science data and, further, to better evaluate the use of quantitative research results. Several advanced statistical and analytic techniques will be introduced and applied to research questions from political science. Participants will leave the course with the ability to better assess published research, better perform their own analyses of quantitative data, and better describe and understand the results of such analyses. 

\vspace{1em}
Among others, the course will discuss the following techniques and topics:

\begin{itemize*}
\item Linear regression analysis
\item Interpretation of interactions in regression models
\item Regression with binary, categorical, and count outcomes
\item Analysis of data gathered over time and across geographies
\item Research design for causal inference 
\item Using Stata for statistical analysis
\end{itemize*}

In addition, topics such as hypothesis generation and research design will be discussed throughout the course. The course does not have a unified theoretical or empirical focus, but we will touch upon and read empirical literature from most areas of political science including comparative politics, public administration, and international relations.

\vspace{1em}
Each week, the course will consist of two sessions:\\

\begin{tabular}{lll}
Lecture & Tuesday, 14:00--17:00 & Building/Room 1330-024\\
Hands-on lab session & Wednesday, 14:00--16:00 & Building/Room 1341-315\\
\end{tabular}


\clearpage
\section{Objectives}
The learning objectives for the course are as follows. By the end of the course, students should be able to:

\begin{enumerate*}
\item Describe politically relevant research questions and hypotheses
\item Evaluate and deduce observable implications from political science theories 
\item Explain statistical procedures and their appropriate usages
\item Apply statistical procedures to relevant research problems
\item Synthesize results from statistical analyses into well-written and well-structured essays
\item Demonstrate how to use Stata for statistical analysis
\end{enumerate*}

\section{Exam and Weekly Assignments}

\subsection{Exam (Only for Master/KA)}
The final exam is a seven-day written assignment analyzing a topic outlined by the instructors using quantitative data. In addition, the students must write four essays during the course on topics defined by the instructors as a prerequisite for the exam.\\

The essays (3-5 pages) must be written in English and are due via email to the appropriate instructor at 12:00 on the dates listed below. The students will receive feedback on the essays and they must all be approved before taking the final exam.

\begin{enumerate*}
\item Essay 1 due February 27 to David % Something on OLS?
\item Essay 2 due March 20 to Thomas % Something on causality/research design?
\item Essay 3 due April 10 to David % Something on panel/multilevel?
\item Essay 4 due May 8 to Thomas % Something on GLM?
\end{enumerate*}

\subsection{For PhD Students}

Students can choose from the following course elements:

\begin{enumerate*}
\item ``Regression with Matrices and MLE'' (1 essay, 2 ECTS)
\item Sessions 1--5 (1 essay, 7 ECTS)
\item Sessions 1--5, plus one additional session (2 essays, 9 ECTS)
\item Sessions 1--5, plus two additional sessions (3 essays, 11 ECTS)
\item Sessions 1--5, plus three additional sessions (3 essays, 13 ECTS)
\item Entire course (4 essays, 15 ECTS)
\end{enumerate*}

PhD students must notify the course instructors of their choice by the end of Session 5.


\clearpage
\section{Reading Material}
The assigned material for the course includes about 2400 pages including several textbooks and empirical research articles, all of which are available online or in the printed course packet. All readings should be completed for the day they are described. {\em There is reading assigned on the first day.} The textbooks for the course are as follows:\\

\reading{CameronTrivedi2010}
\reading{Sonderskov2014}
\reading{AngristPischke2009}
\reading{Long1997}
\reading{Allison2009} % Panel book
\reading{Berry1993} % OLS book

\noindent Note: An English-language translation of the S{\o}nderskov book, with identical content, is also available from the same publisher.\\

\noindent PhD students are additionally required to obtain:\\
\reading{Fox2009}

\section{Software}
The course will use \href{http://www.stata.com/}{Stata 13} for all all in-class activities, lab sessions, homework, and the exam. Stata is not free, but a student license can be purchased for 200kr. Details on purchasing Stata are available here: \href{http://studerende.au.dk/en/selfservice/local-it-services-and-support/it-at-bss/analytics-tools/stata/}{http://studerende.au.dk/en/selfservice/local-it-services-and-support/it-at-bss/analytics-tools/stata/}. Laboratory sessions will meet in Room 1341-315, which has computers with Stata available for free use.\\

Note: PhD students will also use R for the ``Regression with Matrices and MLE'' module. R can be freely downloaded from \href{http://cran.r-project.org/}{http://cran.r-project.org/}.

\section{Course Website}
All information about the course will be posted on Blackboard. Any changes to the syllabus or additional notes will be made available there.





\clearpage
\section{Schedule}
The general schedule for the course is as follows. Details on topics covered and the readings for each week are provided on the following pages.

\secttoc

\clearpage

\subsection{Introduction and Research Design I (Week 6)}
\emph{What topics will we cover in this course? How do we think about causality for the purposes of research design? How can experiments help us understand causal relationships?}

\thomas

\subsection*{Lecture}

\begin{itemize*}
\item Course overview
    \begin{itemize*}
    \item Readings and textbooks
    \item Exam
    \item Four essays
    \item Plan for the course
    \end{itemize*}
\item Asking good research questions
\item Hypothesis generation
\item Philosophies of causality
\item Research design, experiments, and matching
\end{itemize*}

\subsubsection*{Readings}
\textbook[Ch.2 from ]{Angrist and Pischke} %13
\reading{ImaiKingStuart2008}               %21
\reading{Gilens2001}                       %17
\reading{HendersonChatfield2011}           %13

\seealso

\reading{HoImaiKingStuart2007}
\reading{Rubin2001}

\subsection*{Lab}
\begin{itemize*}
\item Basics in Stata
\end{itemize*}

\subsubsection*{Readings for lab}
\textbook[Ch. 1--3 from ]{S{\o}nderskov}   %48






\clearpage
\subsection{Research Design II (Week 7)}
\emph{What does it mean to make a causal inference? Can we make causal inferences from non-experimental data? What are the necessary requirements for causal inference?}

\david

\subsection*{Lecture}

\begin{itemize*}
\item Defining causality in a regression framework
\item Model-building
\item Reporting regression results
\end{itemize*}

\subsubsection*{Readings}
\textbook[Ch.3 (up to p.69) from ]{Angrist and Pischke}  % 43
\reading[Ch.3 from ]{KingKeohaneVerba1994}               % 40
\reading[pp.17--25 from ]{Gujarati2002}                  % 9
\reading{Holland1986}                                    % 16

\seealso

\reading{AshworthEtAl2008}
\reading{Prior2005}

\subsection*{Lab}
\begin{itemize*}
\item Further basics in Stata
\end{itemize*}


\subsubsection*{Readings for lab}
\textbook[Ch. 4--6 from ]{S{\o}nderskov}                  % 96





\clearpage
\subsection{Ordinary Least Squares Regression (Week 8)}
\emph{How do we estimate causal effects using regression analysis? How do we interpret linear regression coefficients for different types of variables? What do goodness-of-fit measures tell us?}

\thomas

\subsection*{Lecture}

\begin{itemize*}
\item OLS method
\item Interpretation of coefficients
\item Standard errors, t-tests, p-values, and confidence intervals
\item Goodness-of-fit measures
\end{itemize*}

\subsubsection*{Readings}
\textbook[pp.1--67 from ]{Berry}                          %67
\textbook[Ch.3 (pp.69--109) from ]{Angrist and Pischke}   %41
\textbook[Ch.3 (up to p.103) from ]{Cameron and Trivedi}  %31
\reading{Englebert2000}                                   %28
\reading{CusackIversenSoskice2007}                        %19

\seealso

\reading{Wooldridge2013}
\reading[Ch.2,4--5 from ]{Freedman2009}


\subsection*{Lab}
\begin{itemize*}
\item OLS in Stata
\item Interpretation of coefficients for continuous and factor variables
\item Reporting regression results
% other standard error stuff?
\end{itemize*}


\subsubsection*{Readings for lab}
\textbook[Ch. 8--9 from ]{S{\o}nderskov}                  %50




\clearpage
\subsection{Ordinary Least Squares Regression II (Week 9)}
\emph{How do we state and test hypotheses about heterogeneous effects? How do we interpret those effects in OLS interaction terms? When do we need to estimate alternative standard errors for OLS estimates?}

\david

\subsection*{Lecture}
\begin{itemize*}
\item Effect heterogeneity and interaction terms
\item Standard errors
\item Heteroskedasticity
%\item Clustered standard errors
\end{itemize*}

\subsubsection*{Readings}
\textbook[pp.~293--315 from ]{Angrist and Pischke}   % 23
\textbook[p.~67 to end from ]{Berry}                 % 24
\reading{Friedrich1982}                             % 37
\reading{BramborClarkGolder2006}                    % 20
\reading{MitchellMoore2002}                         % 15
% clustering?
% application

\seealso

\reading{Rhodes2010}
\reading{GlasgowGolderGolder2011}

\subsection*{Lab}

\begin{itemize*}
\item Estimating and interpreting interaction terms
\item Heteroskedasticity-consistent standard errors
\item Clustered standard errors
\item The \texttt{margins} command
\end{itemize*}

\subsubsection*{Readings for lab}
\textbook[Ch. 10 from ]{S{\o}nderskov}              % 20





\clearpage
\subsection{Practical Data Issues (Week 10)}
\emph{How do we construct, tidy, and transform empirical observations into an organized rectangular dataset for use in our analyses? What problems do we encounter when dealing with real-world data? How do we address those challenges?}

\thomas

\subsection*{Lecture}
\begin{itemize*}
\item Data acquisition
\item Data cleaning and transformations
\item Multivariate scaling and reliability
\item Missing data, case deletion, and imputation
\item Data visualization
\end{itemize*}

\subsubsection*{Readings}
\reading[Sections 8.4--8.9 (pp. 338--356) from ]{Lohr2009}  % 18
\reading{KastellecLeoni2007}                                % 17
% something we can replicate (see Greg Huber)
% data gathering

\seealso

\reading{Allison2002}
\reading{Little1992}
\reading{Wickham2014}
\reading{Tufte1983}


\subsection*{Lab}

\begin{itemize*}
\item Data cleaning, tidying, and transformation
\item Multivariate scaling and reliability
\item Variable transformations in regression
\item Missing data handling
\end{itemize*}

\subsubsection*{Readings for lab}
\textbook[Ch. 2 from ]{Cameron and Trivedi}                 % 40
\textbook[Ch. 7 from ]{S{\o}nderskov}                       % 12


\clearpage
\subsection{Research Designs for Causal Inference (Week 11)}
\emph{What kinds of naturally occurring variation can we utilize to make meaningful causal inferences? What sorts of research designs (and associated analyses) that do not involve researcher-created randomization allow us to infer causal relationships?}

\thomas

\subsection*{Lecture}
\begin{itemize*}
\item Instrumental Variables (IV) regression
\item Regression Discontinuity (RD) designs
\item Difference-in-differences (DID)
%\item Synthetic cohorts
\end{itemize*}

\subsubsection*{Readings}
\textbook[Ch.4,6 from ]{Angrist and Pischke}              % 125
\reading{AcemogluJohnsonRobinson2001}                     % 32
\reading{LassenSerritzlew2011}                            % 21
\reading{CampbellRoss1968}                                % 21

\seealso

\reading{Rosenbaum2002}
\reading{MorganWinship2007}
\reading{ShadishCookCampbell2001}
\reading{AngristImbensRubin1996}
\reading{BertrandDufloMullainathan2004}
\reading{EriksonStoker2011}

\subsection*{Lab}

\begin{itemize*}
\item Instrumental variables
\item Analyzing regression discontinuity designs
\end{itemize*}

\subsubsection*{Readings for lab}
\textbook[Ch. 6 from ]{Cameron and Trivedi}              % 31


\clearpage
\subsection{Panel Analysis for Continuous Outcomes (Week 12)}
\emph{What is panel data? How does the nature of panel data relate to the assumptions of the classic linear regression model? How can we leverage time trends to discuss causation? How can we account for unit heterogeneity in a panel setting?}

\david


\subsection*{Lecture}
\begin{itemize*}
\item Difference-in-differences, continued
\item First-differences and fixed effects
\item Random effects models
\end{itemize*}

\subsubsection*{Readings}
\textbook[Ch.~5 from ]{Angrist and Pischke}  % 27
\textbook[Ch.~1,2 from ]{Allison}            % 27
\reading[Ch.~16 from ]{Gujarati2002}         % 20
\reading{FinkelSmith2011}                    % 19
\reading{GerberHuber2010}                    % 21

\seealso

\reading{Stimson1985}
\reading{Beck2001}
\reading{LassenSerritzlew2011}
\reading{PickeringKisangani2010}
\reading{PlumperTroegerManow2005}

\subsection*{Lab}

\begin{itemize*}
\item Data in ``wide'' and ``long'' formats
\item Panel regression
\end{itemize*}

\subsubsection*{Readings for lab}
\textbook[Ch.8 from ]{CameronTrivedi2010}  % 50


\clearpage
\subsection{Multi-level Modeling (Week 13)}
\emph{What do we do when we have data at more than one level of analysis? How does the nature of multilevel data differ from the assumptions of the classical linear regression model? Can we simultaneously summarize relationships at different levels of analysis?}

\david

\subsection*{Lecture}
\begin{itemize*}
\item Multi-level data structures
\item The logic of multi-level models
\item The hierarchical linear model and its interpretation
\item Random effects versus fixed effects
\item Panel models as special cases of multilevel models
\end{itemize*}

\subsubsection*{Readings}
\reading[pp.~51--83; 91--130 from ]{Rabe-HeskethSkrondal2008}  % 73
\reading[Ch.~1--3 from ]{RaudenbushBryk2002}                  % 65

\seealso

\reading{Sonderskov2011a}
\reading{FieldhouseTranmerRussell2007}
\reading{SteenbergenJones2002}


\subsection*{Lab}

\begin{itemize*}
\item Estimation
\item Testing for fixed versus random effects
\item Interpretation
\end{itemize*}

\subsubsection*{Readings for lab}
\reading[Review previous readings from ]{Rabe-HeskethSkrondal2008}
\textbook[pp.~56--58; 305--316 from ]{Cameron and Trivedi}


\clearpage
\subsection{Maximum Likelihood Estimation (Week 14)} % Easter Th Apr 2 - Mon Apr 6 (Meet in wk 14 if Mon/Tu class; otherwise meet in wk 15)
\emph{What is the likelihood theory of statistical inference? What do we do when our dependent variable is binary? What is the relationship between classical linear regression models and generalized linear models?}

\david

\subsection*{Lecture}
\begin{itemize*}
\item Probability distributions
  \begin{itemize*}
  \item Discrete, continuous
  \end{itemize*}
\item Maximum Likelihood Estimation
\item Generalized Linear Models
\item Logistic regression
\end{itemize*}

\subsubsection*{Readings}
\textbook[Ch.1--3 from ]{Long}     % 84
\reading{Hobolt2007}               % 32
\reading{PriceZaller1993}          % 32


\seealso

\reading[(13--26) from ]{Maddala1986}
\reading[Ch.~1--3 from ]{King1989}
\reading[Ch.~1--4 from ]{Gill2000}
\reading{Ragsdale1984}
\reading{McCarthyMcPhailSmith1996}


\subsection*{Lab}

\begin{itemize*}
\item Estimation of maximum likelihood models
\item Interpretation of logistic regression coefficients
\item Interpret Wald tests and likelihood ratio tests
\end{itemize*}


\subsubsection*{Readings for lab}
\textbook[pp.~459--479 from ]{Cameron and Trivedi}
\reading[pp.~75--128; 157--177 (just skimming sections on probit) from ]{LongFreese2005}  % 108
\textbook[Ch.~11 from ]{Sonderskov2014}                             % 27

\clearpage
\subsection{Interpretation of GLMs (Week 15)}
\emph{How do we interpret the results of models involving binary outcomes? How can we translate regression results for these models into meaningful quantities of interest and interpretable graphical results?}

\thomas

\subsection*{Lecture}
\begin{itemize*}
\item Logit versus Probit models % or maybe this will be covered in previous week
\item Heterogeneous effects and interaction terms
\item Predicted probabilities and marginal effects
\item Interpretation of GLMs
\end{itemize*}

\subsubsection*{Readings}

\reading{Houston2006}                                         % 19

\seealso

\textbook[Ch.4 from ]{Long}
\reading{Hobolt2007}
\reading{AiNorton2003}

\subsection*{Lab}

\begin{itemize*}
\item Predicted probabilities and visualization (e.g., \texttt{marginsplot})
\item \texttt{margins} for generalized linear models
\end{itemize*}

\subsubsection*{Readings for lab}
\textbook[Ch.14 from ]{Cameron and Trivedi}                     % 40


\clearpage
\subsection*{No class (Week 16)} % MPSA


\clearpage
\subsection{GLMs for Ordered, Multinomial, and Count Outcomes (Week 17)}
\emph{What do we do when our dependent variable is qualitative, categorical, or a count? How can we make predictions from qualitative, categorical, or count models? What are the similarities between linear and generalized linear models when it comes to inference?}

\david

\subsection*{Lecture}
\begin{itemize*}
\item Ordered logit and probit
\item Multinomial logit
\item Count outcomes
    \begin{itemize*}
    \item Poisson regression
    \item Dispersion and alternative count models
    \end{itemize*}
\end{itemize*}

\subsubsection*{Readings}
\textbook[Chs.~5,6,8 from ]{Long}     % 107
\reading{WhittenPalmer1996}         % 30

\seealso

\reading{ShieldsHuang1995}
\reading{Michelson2003}
\reading{AlvarezNagler1998}

\subsection*{Lab}

\begin{itemize*}
\item Estimation of ordered, multinomial, and count models
\item Interpretation of results
\item Presentation of results
\item Testing for overdispersion in Poisson models
\end{itemize*}

\subsubsection*{Readings for lab}
\textbook[pp.~491--503; 525--529; 567--577 from ]{Cameron and Trivedi}  %29                     % 40



\clearpage
\subsection{Survival and Duration Analysis (Week 18)}
\emph{What do we do if the quantity that we want to model is time? What if the phenomenon that we want to examine is political change? How can we study the duration and timing of political events? What are the similarities between duration models and other regression models when it comes to estimation and interpretation?}

\david

\subsection*{Lecture}
\begin{itemize*}
\item The logic of survival/duration analysis
\item Maximum likelihood survival models
\item The Cox model
\item The proportional hazards assumption
\item Model diagnostics
\end{itemize*}

\subsubsection*{Readings}
\reading{Box-SteffensmeierJones1997}                            % 48
\reading[pp.~21--31; 37--67 from ]{Box-SteffensmeierJones2004}  % 70
\reading{Box-SteffensmeierZorn2001}                             % 17

\seealso

\reading{Box-SteffensmeierReiterZorn2003}
\reading{Bennett1997}
\reading{Ahlquist2010}
\reading{Cunningham2011}
\reading{SimmonsElkins2004}

\subsection*{Lab}

\begin{itemize*}
\item Estimation of maximum likelihood survival models
\item Model testing and comparison
\item Estimation of Cox models
\item Testing the proportional hazards assumption
\item Interpretation of coefficients
\end{itemize*}

\subsubsection*{Readings for lab}
\reading[pp.~50--78; 122--127; 157--162; 166--167; 178--184; 224--234 from ]{ClevesGouldGutierrez2003}  % 61

\clearpage
\subsection{Panel Analysis for Discrete Outcomes (Week 19)}
\emph{How do we analyze data involving both repeated observation of the same units and non-continuous outcomes?}

\thomas


\subsection*{Lecture}
\begin{itemize*}
\item Panel binary outcome models
\item Panel count models
\end{itemize*}

\subsubsection*{Readings}
\textbook[Ch.3,4 from ]{Allison}                        % 41
\reading{JenningsMarkus1984}                            % 19
\reading{Wawro2001}                                     % 16
\reading{Seeberg2013}                                   % 18


\seealso

\reading[Ch.5 from ]{AndresGolschSchmidt2013}
\reading[Ch.17.4 from ]{Greene2013}

\subsection*{Lab}

\begin{itemize*}
\item Estimating panel models for binary outcomes
\item Model interpretation
\end{itemize*}

\subsubsection*{Readings for lab}
\textbook[pp.615--641 from ]{Cameron and Trivedi}      % 27




\clearpage
\subsection{Conclusion and Wrap-up (Week 20)} % May 13
\emph{What have we learned? What didn't we learn?}

\david
\thomas

\vspace{1em}
\subsection*{Lecture}
\begin{itemize*}
\item Wrap-up
\item Course evaluations
\item Questions about the exam
\end{itemize*}

\subsubsection*{Readings}
\reading{SimmonsHopkins2005}                           % 9
\reading{LauPatelFahmyKaufman2014}                     % 21

\seealso
% time-series
% more hierarchical models
% experiments
% structural equation modelling
% factor analysis and principal componenents analysis
% machine learning: cluster analysis, etc.
% latent class analysis
% bootstrapping
% Bayes
% spatial regression
% sample selection models
% survey design, sampling, weighting
% text analysis
% quantile regression
% non-parametric regression
% data visualization




% load bibtext, but don't generate a bibliography
\bibliographystyle{plain}
\nobibliography{Syllabus}


\end{document}
